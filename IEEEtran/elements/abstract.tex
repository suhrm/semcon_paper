%!TEX root = ../Master.tex
\begin{abstract}
Most video codecs do not tolerate data loss. Therfore, a reliable multicast scheme is preferable for live video streaming from a drone to multiple receivers. We examine how adaptation of video, transmission and erasure coding rate can improve reliability of multicast. Based on the examination we design a joint adaptation policy, for continuous and delay bounded video playback. First, we analyse the characteristics of the wireless channel when using a drone through a measurement campaign. This measurement campaign resulted in traces for different transmission rates specified in 802.11g as well as video traces from a Bebop 2 drone. Second, we leveraged these traces to understand the conditions for our joint design looking not only at the achievable video rate, but also at the inherent delay introduced on a per frame basis. Our results show that transmission rate adaptation does not improve the reliability in the examined test scenario. Changing the amount of redundancy added by erasure coding combined with adaptation of the video rate increase the reliability. We propose two adaptation policies that minimize video playback freezes while maximizing the video quality for both playback methods.


%In order to achieve reliable and adaptive multicast with drones, we characterize the contributions and requirements for source, channel, and network coding of this system. First, we analyse the characteristics of the wireless channel when using drones through a measurement campaign. This measurement campaign resulted in traces for different modulation and coding schemes of 802.11 as well as video traces for a flying Bebop 2 drone. Second, we leveraged these traces to understand the conditions for our joint design looking not only at achievable data rates at the source, but also the inherent delay introduced on a per packet basis. Our results show that rate adaptation does not improve the reliability as a function of distance in the presented test scenario. Whereas the amount of redundancy and source coding could be adjusted to increase the reliability based on the loss reports from the receivers and the delay requirements.


%This paper examines how different reliability methods could be used to perform reliable multicast from a drone in an open field. This includes Source Coding, Erasure Coding and Rate Adaptation. This is done by measuring various metrics of the packets loss at the receivers. This includes how packet loss correlates between multiple receivers,How distance between the receivers and transmitter correlates with the packet loss, how a h264 behaves and how the loss rates are when switching between them while flying. We see that the loss correlation is very small and we suggest using network coding. We see based on our measurements that rate adaptation does not improve the reliability in the case that we have tested. Whereas the amount of redundancy and the Source Coding could be adjusted to increase the reliability based on the loss reports from the receivers and the delay requirements.

\end{abstract}
