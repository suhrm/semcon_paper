%!TEX root = ../Master.tex
\section{Conclusion \& Future work}\label{sec:conc}
Plain 802.11 multicast is inadequate for live video streaming. Additionally, live streaming impose requirements on the video playback delay. Increased reliability for continuous and delay bounded video playback can be achieved by jointly adapting TxR, RLNC and video rate based on PER. Our investigation is based on experimental data obtained while using a drone in an open field, which we analyse using a simple delay model. We show that for this particular case, TxR adaptation yield no improvement on the reliability, thus picking the highest rate is the best choice. Our recommendation is using a policy to adapt RLNC and video rate based on PER. For continuous playback, we propose picking the highest video rate, which introduces the least amount of playback delay. Thereby a high video rate is maintained for the majority of the time while minimizing the playback delay and freeze periods. For delay bounded playback the delay is constant. Here, we choose the highest video rate, which does not exceed the delay bound. Frames delayed beyond the delay bound are dropped. This policy minimize cumulative freeze time and the number of freeze periods.

%. For continuous playback we propose a policy that minimizes the playback delay and freeze periods, while retaining a high video rate. For delay bounded playback we propose a policy which minimize cumulative freeze time and the number of freeze periods, while retaining a high video rate.
%
%When live video streaming from a drone to multiple users a number of components that can be used in order to optimize the experienced video quality at the receiver. We consider TxR, RLNC, source coding and playback delay. Changing data rate did not affect the packet error rate. Thus picking the highest data rate is always the best decision. RLNC is a nice technique, which allows us to deal with the varying packet error rate in a highly adaptable manner, and has additional benefits if the receivers were allowed to communicate, because of the recoding feature. Source coding applies the biggest restraint on the system, because it shows a highly fluctuating bit rate, which is undesirable in relation to the much more constant data rate, which is supported by the system. We show that it is necessary to change source bit rate as the drone flies away from the receiver. When exactly to change the source bit rate is highly dependent on the desired delay of the video. Allowing a 1.1 second delay would accommodate a video bit rates of 5 and 10 Mb/s for the entire trace period. Interestingly, only a few seconds of the 20 Mb/s video has a delay higher than 1.1s, which means that the 20 Mb/s video could be used in all other periods than the few seconds where it can not be supported. When a stricter constraint of the delay is imposed, we see that none of the video bit rates can be used to deliver a lag free video. The number and length of the freezes are very similar between the different video bit rates.
As these policies are based on a simplified model, further investigation is required before applying them in real applications. Real implementation of RLNC would introduce coding delay and overhead. One would have to ensure that when switching video rate the newest frame should be compatible with the last frame transmitted. Furthermore, the policies are only based on feedback from a single receiver. How to weigh the feedback from multiple receivers is still an open problem since this is specific to the scenario. A real feedback system would also introduce delay.% which increases the probability of incorrect adaptation due to mismatch.

	Future investigations should focus on implementing this policy in a test bed, such that the model assumptions are eliminated. This enables investigation of how factors, which are unaccounted for in the current model, effect the performance of the proposed policies. Another interesting aspect is expanding the capabilities of the receivers, e.g. by allowing cooperative sharing among receivers.



%Future work should look into the specifics of a suitable RLNC scheme, further investigate and design a feedback system and look into other scenarios. An implementation would be a logical and a very strong tool to visualize the performance and see the effect/significance of the assumptions/simplifications made in this paper.

%% Allowing "catch-ups" every 10 or 20 or 30 seconds might give a short delay?
